\documentclass[12pt]{article}   % you have 10pt, 11pt, or 12pt options

\setlength{\textwidth}{17.2cm}     % if you change this, consider changing
\setlength{\evensidemargin}{-.3cm} % side margins to retain centering
\setlength{\oddsidemargin}{-.3cm}

\setlength{\textheight}{23cm}   % if you change this, consider changing
\setlength{\topmargin}{-2cm}  % top margin to retain centering
\setlength{\headsep}{1.6cm}

%---------------------- These packages below add functionality to your version of LaTeX --------------
%---------------------- You might not use all of them --------------------------------------
\usepackage{amssymb}
\usepackage{latexsym}
\usepackage{amsthm}
\usepackage{enumerate}
\usepackage{epsfig}
\usepackage{graphicx}
\usepackage{color}
\usepackage{float}
\usepackage{subfigure}
\usepackage{amsmath}
\usepackage{makeidx}
\usepackage{fancyhdr}
\usepackage{lastpage}
\usepackage{url}
\pagestyle{fancy}
\pagestyle{empty}


%---------- the symbols below will give you the blackboard bold of R, T, etc. ----------
\DeclareSymbolFont{AMSb}{U}{msb}{m}{n}  
\DeclareMathSymbol{\Sph}{\mathbin}{AMSb}{"53} \DeclareMathSymbol{\R}{\mathbin}{AMSb}{"52}
\DeclareMathSymbol{\T}{\mathbin}{AMSb}{"54} \DeclareMathSymbol{\Z}{\mathbin}{AMSb}{"5A}
\DeclareMathSymbol{\K}{\mathbin}{AMSb}{"4B}

%------------------------- Theorem and Proof Environments -------------------------------------------------

% This section defines all the environments you might use.  Just type
% \begin{theorem, or corollary, or whatever}, then the optional name of the
% theorem inside {} (or empty {} if no name), then body of the theorem,
% corollary, whatever, also inside {} then \end{theorem, corollary, whatever}
%
% Notice when I use them in the paper, I put an optional "argument" to the function
% and this gives a name to the theorem

\newenvironment{theorem}[1]{\vspace{.9cm}\noindent    {\bf Theorem {#1}}}{\vspace{.1cm}}
\newenvironment{lemma}[1]{\vspace{.9cm}\noindent    {\bf Lemma {#1}}}{\vspace{.1cm}}
\newenvironment{corollary}[1]{\vspace{.9cm}\noindent    {\bf Corollary {#1}}}{\vspace{.1cm}}
\newenvironment{definition}{\vspace{.9cm}\noindent {\bf Definition}}{\vspace{.1cm}}
\def\qed{\hfill $\Box$}
\renewenvironment{proof}{\vspace{.5cm}   \noindent{\bf Proof: }}{\qed \vspace{1cm}}
 

%Picture inclusion

\newcommand\pic[3]{
\begin{figure}[H] \begin{center} 
\epsfig{file=#1, height=#2pt} 
\end{center} 
\caption{#3} 
\end{figure}
}

\def\inj{\text{inj}}
\def\diam{\text{diam}}
\def\area{\text{area}}
\def\length{\text{length}}

% Keywords command
\providecommand{\keywords}[1]
{
  \small	
  \textbf{\textit{Keywords---}} #1
}

%*****************************************************************************************

\begin{document}  % necessary part of document

\title{  LaTeX \, Template and Tutorial for Math Modelers  }

\author{Your Name\\
        \small{ Your Department, Your University }\\
        \small{ E-mail: Your E-mail address }
        }
        
\date{} %No Date

\maketitle

\thispagestyle{empty}

\begin{abstract}
This is a template of the abstract of the contributing presentation for 2020 TMS Annual Meeting. The file should be named such as
 \begin{center}{\tt T[No.]\textunderscore [Name of the Author]\textunderscore Abstract}.\end{center}
 \noindent The number in the bracket represents the topic. Please refer to the website of 2020 TMS for the number of the topic.\\
\end{abstract}

\keywords{   keyword 1, keyword 2, keyword 3    }

%------------------------------------------------------------------------------------
\begin{thebibliography}{99}  % 99 is the highest number of references this expects;  
                                                       % this sets the expected width of the citation
\bibitem{ } 
P. Erd\H os, 
\emph{ A selection of problems and results in combinatorics },  
Recent trends in combinatorics (Matrahaza, 1995),  
Cambridge Univ. Press, 
Cambridge, 2001, pp. 1--6.

\bibitem{ }
R.L. Graham, D.E. Knuth, and O. Patashnik, 
\emph{Concrete mathematics}, 
Addison-Wesley, 
Reading, MA, 1989.

\bibitem{ } 
D.E. Knuth, 
\emph{   Two notes on notation  }, 
Amer. Math. Monthly \textbf{99} (1992), 403--422.

\bibitem{ } 
H. Simpson, 
\emph{Proof of the Riemann Hypothesis},  
preprint (2003), available at 
\url{http://www.math.drofnats.edu/riemann.ps}.

\end{thebibliography}

\end{document}

